%
\comment{

1. Benefit of OpenFlow's flow table architecture: being able to reconfigure on the fly
2. Performing full FPGA reconfiguration cannot be done on the fly
3. Partial reconfiguration could solve this, but has yet to be proven to be feasible
4. The NoC is more amenable to partial reconfiguration, as modules are decoupled (don't have to have same clock either)
5. Partial reconfiguration of individual processing modules has the potential to be done on the fly
6. Left for future work

}
%

A key advantage to OpenFlow's flow table architecture is the capability of updating processing rules ``on-the-fly''. In other words, the table entries can be updated with new match-action rules even while the processor is receiving packets.
This is trickier in the NoC-PP architecture.
%Modern FPGAs require the reconfiguration of the entire chip in order to perform a design change.
%During reconfiguration, the chip's transceivers cannot accept any data.
Although NoC-PP may still contain tables that can perform soft updates, its decreased reliance on memory means that more of its processing has been moved to dedicated logic in the FPGA fabric.
Typically, modifying an FPGA design requires complete reconfiguration of the device, during which the transceivers cannot accept any data.
Making processing rule updates through a complete reconfiguration would require the packet processor to be effectively ``paused''.

Partial reconfiguration~\cite{kao2005benefits} presents a potential solution to this limitation.
%Unlike the monolithic PP design, which would require a complete reconfiguration to perform architecture changes, NoC-PP has both logically and physically partitioned processing modules.
Unlike the monolithic PP design, NoC-PP has both logically and physically partitioned processing modules.
The modules share a common interface and are decoupled by the NoC, making the design highly amenable to partial reconfiguration of individual processing modules.
While soft flow table updates allows for processing rule changes, partial reconfiguration makes both rule and architecture changes possible.
To perform any architecture changes to PP, such as modifying table sizes or the action set, a complete reconfiguration would be necessary.
On the other hand, NoC-PP can partially reconfigure individual processing modules while the remaining modules remain operational.
Thus, partial reconfiguration would open up the possibility of modules being updated/added/removed while the packet processor remains ``live''.
%This would open up the possibility of modules being updated/added/removed while the packet processor remains ``live''.
%However, its feasibility has been met with various challenges, such as \textit{blank, blank and blank}.
%Using the embedded NoC to decouple a design such as NoC-PP has the potential to facilitate partial reconfiguration.
An investigation into partial reconfiguration in a NoC-enhanced FPGA is beyond the scope of this paper and is left for future work.

