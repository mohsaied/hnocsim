%
\comment{

1. Use the embedded NoC as the crossbar!!!
2. NoC flow control
3. NoC characteristics (radix, topology, link width, frequency, BW)
4. How to use NoC as crossbar
5. How to connect more than just 16 modules (arbiter design and block diagram)

}
%

To address this problem, we draw inspiration from previous work that used an embedded NoC in an FPGA as a crossbar for an Ethernet switch~\cite{bitar2014efficient,abdelfattah2015take}.
This NoC -- described in Section~\ref{sec:noc-fpga} -- can function as the full crossbar in our packet processor design.
Not only is it already designed to transfer packets across the FPGA, it also includes a built-in flow control mechanism that can handle scenarios of adversarial traffic, such as prolonged bursts of packets.
For example, if a certain processing module is busy and cannot accept packets from the NoC, the NoC will hold packets destined for that module in a buffer at the connecting router.
Should that buffer become full, packets will then be buffered at downstream NoC routers.
If the NoC cannot accept packets at one of its routers due to a buffer being full, then it can also send a backpressure signal to the modules connected to that router.

%\hl{consider removing this text in favour of embedded NoC section}
%The NoC used in our design is a 16-node mesh with 150b-wide links and capable of running at 1.2 GHz in 28-nm process technology (the same process used for Stratix-V devices).
With a capacity of 180 Gb/s at each of its links, the NoC can transport high bandwidth data throughout the FPGA, to and from processing modules in our design.
Processing modules are connected to NoC routers, as illustrated in Figure~\ref{noc-parser-general}, with the NoC's FabricPort used to bridge the frequency of the processing modules and the frequency of the NoC (see Section~\ref{sec:noc-fpga}).
%In effect, this design uses the embedded NoC to replace the full crossbars in Figure~\ref{high-level-pp}.
Moreover, we can combine multiple modules at a single router node using the FPGA's soft logic for arbitration.
For the remainder of the paper, we shall refer to this NoC packet processor as ``NoC-PP''.

%In order to not be limited to a maximum of 16 processing modules, we created arbitration logic to connect multiple modules to a single NoC router (Figure~\ref{arbiter}).
%To maximize independence between each module connected to the router, FIFO queues were added between each module and the arbitration logic.
%This allows for each module to independently send and receive backpressure to and from the NoC.
%The queue depth depends on the number of modules connected and the desired degree of independence between modules.

%
\figvs{1}{noc-parser-general}{}{The ``NoC-PP'' architecture: the embedded NoC serves as the crossbar for the module-based packet processor design. NoC routers and links are ``hard'', i.e. embedded in the chip, while the processing modules and arbitration logic are synthesized from the FPGA ``soft'' fabric.}
%

%
%\figvs{1}{arbiter}{}{Arbitration logic for connecting multiple processing modules to a single NoC router.}
%
