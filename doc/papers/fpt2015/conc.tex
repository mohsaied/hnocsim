%
%
%
%


As network performance becomes ever more crucial to data processing and mobile applications, there is an ever-increasing demand on network infrastructures to evolve.
The NoC-PP packet processor architecture proposed in this work focuses on maximizing hardware flexibility, thus providing a platform that is very amenable to network evolution.
By using an FPGA augmented with an embedded NoC, we have proposed a modular packet processor design that is capable of being reprogrammed to support different combinations of actions and header field matches.
Unlike ASIC-based OpenFlow architectures, it is not limited by the action set and table sizes fixed upon chip fabrication.
Its flexibility significantly exceeds that of the RMT ASIC architecture, while matching, and even surpassing, its supported bandwidth (400G/800G vs. 640G).
Compared to the best previously proposed FPGA-based packet processor, NoC-PP proves to be 1.7$\times$ and 3.2$\times$ more resource efficient, and achieves 1.5$\times$ and 3.7$\times$ lower latency on complex and simple applications, respectively.

The NoC-PP architecture is only possible given the inclusion of a hardened NoC embedded into an FPGA.
This hard NoC is both fast and small, enabling easy cross-chip communication.
Though an embedded NoC has yet to be adopted in FPGAs, we hope that the compelling applications explored in this and previous work~\cite{abdelfattah2015take,bitar2014efficient} present a convincing case for its inclusion in future FPGA devices.
Such a NoC-enhanced FPGA could revolutionize SDN by paving the way for a fully-programmable data plane.
