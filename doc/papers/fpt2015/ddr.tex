
\comment{
%
\begin{itemize}
\item Short section saying how we can use DDR - would be stronger if I can run some experiments to quantify when it starts being a net win.
\item Look into video over IP as a sample use-case that is related to networking - does it fit in with openflow?
\item Talk about storage controller and hashing scheme (briefly).
\item Say how it could fit in (store the body and route the header then fetch the body back).
\end{itemize}
%
}

In NoC-PP, we stream our data packets through the processor in a fully pipelined way; therefore, we do not need to buffer the packet payloads during processing.
However, other applications may necessitate the storage of packets, such as if the output data rate does not match the input data rate.
One such example is ``video over IP'' where video pixels are transmitted over IP protocol~\cite{altera_voip}.
In such cases, we often only need the packet header for immediate processing, while the packet payload is transmitted unchanged.

We can use our embedded NoC to connect to the FPGA's memory interfaces and use a tagging system to store packet payloads off-chip while the header is being processed.
At transceiver inputs, each packet is assigned a tag which is attached to both the packet header (which remains on-chip) and payload (that is stored off-chip).
A simple storage controller at the external memory interface stores each tag and its external memory address.
After the header is done processing, it requests the payload data from external memory, which is re-joined with the header at the output transceiver.
We aim to better explore designs that require such storage in our future work.
