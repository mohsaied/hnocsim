
\comment{
%
\begin{itemize}
\item NoCs are a form of scalable interconnect that performs distributed arbitration between modules to send data between them.
\item They have built-in arbitration, switching and buffering to be able to do so efficiently.
\item Additionally, built-in credit-based backpressure manages the flow of data packets through the NoC.
\item Previous work has looked into how to include an NoC on FPGAs.
\item Efficient way to do it is to harden the NoC completely and run it at a very high frequency.
\item This frequency is much higher than the FPGA fabric, so to be able to connect an FPGA module, we use a Fabricport that bridges the speed and width of data. 
\item Explain briefly how it works.
\end{itemize}
%
}

\figvs{1}{embedded_noc}{}{A NoC-Enhanced FPGA. The embedded NoC is implemented in hard logic, and connects to modules on the FPGA through a FabricPort~\cite{abdelfattah2015take}.}

Existing FPGAs contain only fine-grained programmable blocks (lookup-tables, registers and multiplexers) from which an application's interconnect can be constructed.
Previous work has proposed the inclusion of an embedded NoC to augment this fine-grained interconnect so as to better interconnect an FPGA application. 
Embedded NoCs have been shown to improve both area and power efficiency, and ease timing closure compared to traditionally-used soft buses configured from the FPGA's programmable blocks~\cite{trets,tvlsi,micro}.

Figure~\ref{embedded_noc} shows an embedded NoC on the FPGA.
The NoC routers and links run approximately four times as fast as the FPGA fabric; therefore, we use a special component called the FabricPort to bridge both width and frequency in a flexible way between the embedded NoC routers and soft FPGA modules~\cite{abdelfattah2015take}.
Furthermore, the processing modules connected to the NoC need not operate using the same clock frequency or phase -- the FabricPort essentially decouples the modules connected together through the NoC.
The routers are packet-switched virtual-channel routers that perform distributed arbitration and switching of packets coming from modules connected to the NoC.
These routers also contain built-in buffering and credit-based backpressure, and so can automatically respond to bursts of data and heavy traffic on its links while maintaining application correctness.

The NoC we use has been evaluated on a large 28-nm Stratix V FPGA, and can run at 1.2~GHz in that process technology~\cite{abdelfattah2015take}.
The NoC links are 150 bits wide each and so can transport up to 180~Gb/s in each direction between any two routers, of which we have 16 as illustrated in Figure~\ref{embedded_noc}.
Because it is implemented in efficient hard logic, this NoC only consumes 1.3\% of the core area of an Altera Stratix V FPGA.
The FabricPort for this NoC can connect a module running at any FPGA frequency and any width between 1 and 600~bits to the NoC routers at 150 bits and 1.2~GHz.
It does so using a combination of clock-crossing and width adaptation circuitry that formats data into NoC packets in an efficient way~\cite{abdelfattah2015take}.




