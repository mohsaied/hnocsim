%
\comment{

1. The selected embedded NoC contains two Virtual Channels, as previous work has shown it improves performance (?) by 30\%
2. Virtual channels are...
3. With two virtual channels, data flowing through the packet processor can be sorted into two groups of traffic
4. This can be useful for establishing a packet priority scheme
5. Packet priority is prevalent in many different modern network protocols, including VLAN and TCP
6. Priority packets can have their own reserved VC, while all other traffic is routed through the other VC

}
%


Virtual Channels (VC) in a NoC are separate FIFO buffers located at every router port.
They allow packets arriving at or being sent along a common physical link to be stored in separate buffers.
The embedded NoC used in our design employs two VCs, as previous work has shown that NoC congestion is reduced by $\sim$30\% when a second VC is used~\cite{fpl}.
With two VCs, data flowing through the packet processor can be sorted into two groups of traffic.
This can be especially useful for establishing a packet priority scheme.
Packet priority is prevalent in many different modern network protocols, including VLAN and TCP.
In NoC-PP, one VC can be reserved for packets given priority based on the contents of their headers, while all other traffic is routed along the other VC.
Thus, priority packets can bypass non-priority packets when the NoC is congested, thereby giving priority packets a faster path through the design.


