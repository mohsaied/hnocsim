%
%
\comment{
\hl{change abstract to reflect new material}
Embedded logic and I/O interfaces have made field-programmable gate-arrays (FPGAs) more capable platforms for implementing large systems.
We explore the addition of a fast embedded network-on-chip (NoC) to augment the FPGA's existing wires and switches, and help interconnect large applications.
A flexible interface between the FPGA fabric and the embedded NoC allows modules of varying widths and frequencies to transport data over the NoC.
We study both latency-insensitive and latency-sensitive design styles and present the constraints for implementing each type of communication on the embedded NoC.
%
%
By augmenting field-programmable gate-arrays (FPGAs) with embedded computation, memory and I/O elements, they have become an efficient platform for compute acceleration and networking applications.
However, implementing on-chip communication is still a designer's burden where custom system-level communication buses are implemented using the fine-grained FPGA logic and interconnect fabric.
We propose augmenting FPGAs with an embedded network-on-chip (NoC) to implement system-level communication.
We design custom interfaces to connect a conventional packet-switched NoC to the FPGA fabric and I/Os in a configurable and efficient way.
We then define the necessary rules and constraints to implement FPGA design styles correctly and efficiently using an embedded NoC -- this lays the foundations upon which we can implement applications using a NoC-enhanced FPGA.
In the second half of this paper, we present four application case studies that highlight the advantages of using an embedded NoC.
We show that access-latency to external memory can be \til1.5\xx lower.
Our application case study with image compression shows that an embedded NoC improves frequency by 10--80\%, reduces utilization of scarce long wires by 40\% and makes design easier and more predictable.
Additionally, we leverage the embedded NoC in creating a programmable Ethernet switch that can support up to 819~Gb/s compared to previous work that only demonstrated 160~Gb/s.
Finally we design a 400~Gb/s packet processor based on our embedded NoC, that is more flexible and efficient compared to other packet processor designs.
}
%
%By augmenting field-programmable gate-arrays (FPGAs) with embedded computation, memory and I/O elements, they have become an efficient platform for compute acceleration and networking applications.
Field-programmable gate-arrays (FPGAs) have evolved to include embedded memory, high-speed I/O interfaces and processors, making them both more efficient and easier-to-use for compute acceleration and networking applications.
However, implementing on-chip communication is still a designer's burden wherein custom system-level buses are implemented using the fine-grained FPGA logic and interconnect fabric.
Instead, we propose augmenting FPGAs with an embedded network-on-chip (NoC) to implement system-level communication.
We design custom interfaces to connect a packet-switched NoC to the FPGA fabric and I/Os in a configurable and efficient way and then define the necessary conditions to implement common FPGA design styles with an embedded NoC.
Four application case studies highlight the advantages of using an embedded NoC.
We show that access latency to external memory can be \til1.5\xx lower.
Our application case study with image compression shows that an embedded NoC improves frequency by 10--80\%, reduces utilization of scarce long wires by 40\% and makes design easier and more predictable.
Additionally, we leverage the embedded NoC in creating a programmable Ethernet switch that can support up to 819~Gb/s -- 5\xx more switching bandwidth and 3\xx lower area compared to previous work.
Finally, we design a 400~Gb/s NoC-based packet processor that is very flexible and more efficient than other FPGA-based packet processors.
%
