
\comment{
The conventional way to do this entails RTL hardware simulations using a simulator like Mentor Graphics' Modelsim.
Designs are entered to these simulators using a HDL such as Verilog, SystemVerilog or VHDL.
Therefore, we would need an HDL version of our NoC and FabricPort to properly simulate an embedded NoC.
Furthermore, the HDL implementation of the NoC would have to be parametrizable (to be able to try out different NoCs) and fully verified (to avoid errors).
We first tried borrowing such an open-source implementation~\cite{becker_router}, but we quickly found it very hard to use.
Other than the intermittent bugs that we discovered, it was very challenging to properly match the interface of the NoC with our FabricPort.
This is because every time we changed an NoC parameter, the packet format became noticeably different.
}

To measure application performance, in terms of latency and throughput, we need to perform cycle-accurate simulations of hardware designs that use an NoC.
However, full register-transfer level (RTL) simulations of complex hardware circuits like NoCs are error-prone and slow.
Instead of using a hardware description language (HDL) implementation of the NoC, we used a cycle-accurate software simulator of NoCs called Booksim~\cite{booksim}.
This is advantageous because Booksim provides the same simulation cycle-accuracy, but runs faster than an HDL model of the NoC, and supports more NoC variations.
Additionally, we are able to define our own packet format which greatly simplifies the interface to the NoC.
Finally, it is much easier to extend Booksim with a new router architecture or special feature, making it a useful research tool in fine-tuning the NoC architecture.

%
\figvs{1}{rtl2booksim}{}{\rtlbook{ }allows the cycle-accurate simulation of an NoC within an RTL simulation in Modelsim.}
%

Booksim conventionally simulates an NoC from a trace file that describes the movement of packets in an NoC.
However, our FabricPort and applications are written in Verilog (an HDL).
How do we connect our hardware Verilog components to a software simulator such as Booksim?
We developed \rtlbook{ } to interface HDL designs to Booksim; Fig.~\ref{rtl2booksim} shows some details of this interface.

The Booksim Interface is able to send/receive flits and credits to/from the NoC modeled by the Booksim simulator through Unix sockets.
Next, there is an RTL Interface that communicates with our RTL HDL design modules.
The RTL Interface communicates with the Booksim Interface through a feature of the SystemVerilog language called the direct programming interface (DPI).
SystemVerilog DPI basically allows calling software functions written in C/C++ from within a SystemVerilog design file.
Through these two interfaces -- the Booksim Interface and the RTL interface -- we can connect any hardware design to any NoC that is modeled by Booksim.
%As Fig.~\ref{rtl2booksim} shows, we can configure the NoC using a configuration file.
%This Booksim configuration file is very easy to use since it simply consists of a list of NoC parameters.
%We made sure to have a simple HDL NoC wrapper so that it acts exactly like an HDL NoC in an RTL simulation.
%The NoC wrapper contains data, ready and valid signals for each router input and output and is therefore straightforward to use.
\rtlbook{ }is released as open-source and available for download at: \url{http://www.eecg.utoronto.ca/~mohamed/rtl2booksim}.
The release includes push-button scripts that correctly start and end simulation for example designs using Modelsim and \rtlbook.
